\chapter{Testing}
\label{sec:testing}
\INITIAL{T}{here are many scenarios} in which MapReduce can be applied. Because this work is meant to applicable to any MapReduce job, tests have been selected in order to cover a varied range of analysis and data types. In this case, the analyses chosen attempt to cover the major MapReduce pattern categories as presented in the text "MapReduce Design Patterns" \cite{Miner2012}. In addition to this, the unique features of Flink are applied in order to establish that non-generic cases are also covered. Of course, in addition to the analysis itself the type of data being visualized is key. As such, these design patterns and features are applied to a varied array of data sources which necessitate the use of all of the most vital data visualization approaches.

%%%%%%%%%%%%%%%%%%%%%%%%%%%%%%%%%%%%%%%%%%%%%%%%%%%%%%%%%%%%%%%%%%%%%%%%%%%%
%%%%%%%%%%%%%%%%%%%%%%%%%%%%%%%%%%%%%%%%%%%%%%%%%%%%%%%%%%%%%%%%%%%%%%%%%%%%
\section{Summarization}
\label{sec:summarization}
\INITIAL{T}{his portion} will eventually include models etc.

\section{Filtering}
\label{sec:filtering}

\section{Data Organization}
\label{sec:dataorganization}

\section{Joins}
\label{sec:joins}

\section{Meta Patterns}
\label{sec:metapatterns}

