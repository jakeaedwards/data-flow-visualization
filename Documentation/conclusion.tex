\chapter{Conclusions}
\label{sec:conclusions}
\INITIAL{T}{he application of visualizations} to in-situ analysis scenarios is still a fledgling area of research, in which this work proposes some initial solutions to the most general of problems. 

\paragraph{Visualizations}
Long before analysis was performed using distributed data analysis platforms, there have been formalized ideas about data formats and appropriate solutions for handling these formats. We have seen that for each of the most commonly encountered types of data in an exploratory analysis, there are basic visualizations which have been widely applied and deemed effective for most purposes. Though this of course leaves room for adjustment of visualization strategy when seeking specific results, in the realm of in-situ processing achieving quick insight with minimal design is ultimately the most efficient strategy. In addition, the more complex a data set is, the more likely that it can be broken into constituent parts or subsets which can then be fitted to different visualizations; thus allowing for multiple visual analyses to be presented for a single data set and thereby increasing the odds of extracting meaningful results. 

\paragraph{Patterns}
On top of specific visualization types being paired with data formats, we examined design patterns in analysis applications which each seemed to be well suited to a specific type of visualization technique. This was roughly analogous to the data format examination in that more complex patterns often yield data sets which are summarized or modified by another, simpler pattern and therefore can have multiple categories of visualization applied when seen in a greater context. Because these patterns represent some of the most common analysis tasks, and each is well suited for visualizations of some form, we have seen that visualizations can yield useful results in most straightforward analysis jobs.

\paragraph{Applications}
Bringing together the connections we drew previously from data, to visualizations, to patterns, we have seen a number of concrete examples of data sets being transformed in different ways and visualized at different steps. In each case, some insight into the data was drawn and the implications for the user of such an analysis task were demonstrated. In some cases visualizations proved to immediately identify important results of an analysis, while in others they identified the steps which should be taken (or not taken) in further analyzing the data by identifying which methods would likely be effective and which would not.

\paragraph{Summary}
Though areas such as automation of visualization and improvements to the presentation of the generated charts could provide great benefit in the future, even the addition of simple general purpose visualization classes such as those implemented in this work provide clear benefit for developers. The introduced overhead to performance is negligible, with linear time execution in almost all cases; and developers only need to add a small amount of code to pre-existing analysis jobs to utilize the visualization capabilities presented. In discovering errors in data flows, unexpected features of data sets, and identifying methods of further analysis, visualization of data sets within data flow graphs provides significant benefit to developers with little to no cost.